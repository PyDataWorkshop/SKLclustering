\subsection{Using MiniBatch KMeans to handle more data}
KMeans is a nice method to use; however, it is not ideal for a lot of data. This is due to
the complexity of KMeans. This said, we can get approximate solutions with much better
algorithmic complexity using KMeans.
\subsubsection{Getting ready}
MiniBatch KMeans is a faster implementation of KMeans. KMeans is computationally very
expensive; the problem is NP-hard.
However, using MiniBatch KMeans, we can speed up KMeans by orders of magnitude. This
is achieved by taking many subsamples that are called MiniBatches. Given the convergence
properties of subsampling, a close approximation to regular KMeans is achieved, given good
initial conditions.
\subsubsection{I@mplementation}
Let's do some very high-level profiling of MiniBatch clustering. First, we'll look at the overall
speed difference, and then we'll look at the errors in the estimates:
\begin{framed}
\begin{verbatim}
>>> from sklearn.datasets import make_blobs
>>> blobs, labels = make_blobs(int(1e6), 3)
>>> from sklearn.cluster import KMeans, MiniBatchKMeans
>>> kmeans = KMeans(n_clusters=3)
>>> minibatch = MiniBatchKMeans(n_clusters=3)
\end{verbatim}
\end{framed}
%========================================================%
% % Building Models with Distance Metrics
% % 98
Understand that these metrics are meant to expose the issue.
Therefore, great care is taken to ensure the highest accuracy of the
benchmarks. There is a lot of information available on this topic;
if you really want to get to the heart of why MiniBatch KMeans is
better at scaling, it will be a good idea to review what's available.
Now that the setup is complete, we can measure the time difference:
>>> %time kmeans.fit(blobs) #IPython Magic
CPU times: user 8.17 s, sys: 881 ms, total: 9.05 s Wall time: 9.97 s
>>> %time minibatch.fit(blobs)
CPU times: user 4.04 s, sys: 90.1 ms, total: 4.13 s Wall time: 4.69 s
There's a large difference in CPU times. The difference in clustering performance is shown
as follows:
>>> kmeans.cluster_centers_[0]
array([ 1.10522173, -5.59610761, -8.35565134])
>>> minibatch.cluster_centers_[0]
array([ 1.12071187, -5.61215116, -8.32015587])
The next question we might ask is how far apart the centers are:
>>> from sklearn.metrics import pairwise
>>> pairwise.pairwise_distances(kmeans.cluster_centers_[0],
minibatch.cluster_centers_[0])
array([[ 0.03305309]])
This seems to be very close. The diagonals will contain the cluster center differences:
>>> np.diag(pairwise.pairwise_distances(kmeans.cluster_centers_,
minibatch.cluster_centers_))
array([ 0.04191979, 0.03133651, 0.04342707])
\end{verbatim}
\end{framed}
\subsubsection{Operation}
The batches here are key. Batches are iterated through to find the batch mean; for the next
iteration, the prior batch mean is updated in relation to the current iteration. There are several
options that dictate the general KMeans' behavior and parameters that determine how
MiniBatch KMeans gets updated.
%========================================================%
% % Chapter 3
% % 99
The batch_size parameter determines how large the batches should be. Just for fun, let's
run MiniBatch; however, this time we set the batch size the same as the dataset size:
>>> minibatch = MiniBatchKMeans(batch_size=len(blobs))
>>> %time minibatch.fit(blobs)
CPU times: user 34.6 s, sys: 3.17 s, total: 37.8 s Wall time: 44.6 s
\end{verbatim}
\end{framed}
Clearly, this is against the spirit of the problem, but it does illustrate an important point.
Choosing poor initial conditions can affect how well models, particularly clustering models,
converge. With MiniBatch KMeans, there is no guarantee that the global optimum will
be achieved.
\newpage
